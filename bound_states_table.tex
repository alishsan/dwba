\documentclass[prb,reprint,amsmath,amssymb,showpacs]{revtex4-1}

\usepackage{graphicx}
\usepackage{dcolumn}
\usepackage{bm}
\usepackage{hyperref}
\usepackage{booktabs}

\begin{document}

\title{From Transcendental Roots to Numerical Benchmarks: High-Precision Bound States of the Finite Square Well}

\author{Your Name}
\affiliation{Your Institution/Department}
\date{\today}

\begin{abstract}
We present a high-precision reference dataset for bound state energies in a finite square well, specifically parameterized for the $^{14}$C + n system. By utilizing a Newton-Raphson root-finding algorithm with analytical derivatives and a transcendental matching error function, we achieve numerical convergence with matching errors as low as $10^{-9}$. These results serve as a pedagogical and technical benchmark for validating numerical Woods-Saxon potential solvers in the limit of small diffuseness.
\end{abstract}

\maketitle

\section{Introduction}
The finite square well is a fundamental model in quantum mechanics pedagogy. While introductory textbooks typically focus on graphical solutions to the transcendental equations, modern computational nuclear physics—such as Distorted Wave Born Approximation (DWBA) calculations—requires high-precision eigenvalues to verify numerical integrators. This paper provides a rigorous reference table for these eigenvalues, parameterizing the system via the dimensionless well depth $z_0$.
\section{Methodology}

\subsection{Dimensionless Parameters and Matching Condition}
The bound states of a particle in a 3D spherical finite well of radius $a$ and depth $V_0$ are determined by matching the interior and exterior wavefunctions at the boundary. We parameterize the system using the following dimensionless quantities:
\begin{align}
\xi &= ka \quad \text{(Inside dimensionless wavenumber)} \\
\eta &= \kappa a \quad \text{(Outside dimensionless decay constant)}
\end{align}
The well depth is characterized by the parameter $Z_0$, defined such that:
\begin{equation}
Z_0^2 = \xi^2 + \eta^2 = \frac{2\mu V_0 a^2}{\hbar^2}
\end{equation}
For a given angular momentum $L$, the continuity of the logarithmic derivative of the wavefunction leads to the transcendental matching condition:
\begin{equation} \label{eq:matching_condition}
\frac{\xi j_L(\xi)}{j_{L-1}(\xi)} + \frac{\eta k_L(\eta)}{k_{L-1}(\eta)} = 0
\end{equation}
where $j_L$ are the spherical Bessel functions of the first kind and $k_L$ are the modified spherical Bessel functions of the second kind.

\subsection{Root-Finding Algorithm}
To find the energy eigenvalues $E$ that satisfy Eq.~(\ref{eq:matching_condition}), we define the objective function:
\begin{equation}
F(E) = \frac{\xi j_L(\xi)}{j_{L-1}(\xi)} + \frac{\eta k_L(\eta)}{k_{L-1}(\eta)}
\end{equation}
While a robust bracketing algorithm like \textbf{Bisection} can be used to locate roots initially, we implement the \textbf{Newton-Raphson method} to achieve high-precision convergence. [cite_start]This requires the evaluation of the derivative $dF/dE$[cite: 1].

\subsubsection{Derivatives via Recurrence Relations}
We derive the analytical derivatives using the recurrence relations for spherical Bessel functions. The derivative of the individual functions is given by:
\begin{align}
\frac{d}{dx} j_L(x) &= j_{L-1}(x) - \frac{L+1}{x} j_L(x) \\
\frac{d}{dx} k_L(x) &= -k_{L-1}(x) - \frac{L+1}{x} k_L(x)
\end{align}


To compute the derivative of the matching terms efficiently, we define the ratio $R_L(\xi) = j_L(\xi) / j_{L-1}(\xi)$. The derivative of the term involving $\xi$ is derived as follows:
\begin{equation}
\frac{d}{d\xi} \left( \xi \frac{j_L(\xi)}{j_{L-1}(\xi)} \right) = 1 - (2L-1)R_L(\xi) + \xi [R_L(\xi)]^2
\end{equation}
A similar identity applies to the exterior term involving $k_L(\eta)$.

\subsubsection{Newton-Raphson Update}
Using the chain rule $\frac{dF}{dE} = \frac{dF}{d\xi}\frac{d\xi}{dE} + \frac{dF}{d\eta}\frac{d\eta}{dE}$, we calculate the slope of the function at the current energy guess. The iterative update is given by:
\begin{equation}
E_{\text{new}} = E_{\text{old}} - \frac{F(E)}{F'(E)}
\end{equation}
The magnitude of the derivative $F'(E)$ provides physical insight into the \textbf{sensitivity} of the state. A small derivative near a root implies a loosely bound state (such as a $1p_{1/2}$ state near threshold), requiring higher precision in the ``shooting'' method to maintain convergence.

\section{Reference Results}
The following tables provide converged values across a range of $z_0$ values. These results are intended to be used as a cross-check for numerical solvers using a Woods-Saxon potential with small diffuseness ($a_0 \ll R_0$).

\begin{table}[h]
\caption{Dimensionless depth $z_0$ for $^{14}$C + n at $R_0 = 2.0$ fm.}
\centering
\begin{tabular}{ccc}
\hline\hline
$V_0$ (MeV) & $z_0$ \\ \hline
30.0 & 2.31 \\
50.0 & 2.98 \\
70.0 & 3.53 \\
\hline\hline
\end{tabular}
\end{table}

\begin{table}[h]
\caption{Detailed Bound State Parameters for $l=0$.}
\centering
\begin{tabular}{c c c c c c}
\hline\hline
$z_0$ & State & $e$-ratio ($|E|/V_0$) & $\xi$ ($ka$) & $\eta$ ($\kappa a$) & Error \\ [0.5ex] 
\hline
2.0  & 1 & 0.101775 & 1.895494 & 0.638045 & -6.78e-09 \\
3.0  & 1 & 0.422976 & 2.278863 & 1.951098 & -7.07e-09 \\
5.0  & 1 & 0.037131 & 4.906295 & 0.963467 & -3.60e-08 \\
     & 2 & 0.730486 & 2.595739 & 4.273422 & -3.59e-08 \\
10.0 & 1 & 0.290496 & 8.423204 & 5.389771 & -6.92e-08 \\
15.0 & 1 & 0.076235 & 14.416907 & 4.141592 & -1.42e-07 \\
\hline\hline
\end{tabular}
\end{table}

\section{Conclusion}
By bridging the gap between analytical textbook models and complex numerical research tools, this dataset provides a rigorous standard for benchmarking nuclear potential solvers. Utilizing Newton-Raphson methods with analytical derivatives allows for a level of precision that makes these tables a reliable  standard for student and researcher alike.

\begin{thebibliography}{9}

\bibitem{Griffiths}
D. J. Griffiths and D. F. Schroeter, \textit{Introduction to Quantum Mechanics}, 3rd ed. (Cambridge University Press, 2018).

\bibitem{Flugge}
S. Flügge, \textit{Practical Quantum Mechanics} (Springer-Verlag, Berlin, 1971).

\bibitem{NumericalRecipes}
W. H. Press, S. A. Teukolsky, W. T. Vetterling, and B. P. Flannery, \textit{Numerical Recipes: The Art of Scientific Computing}, 3rd ed. (Cambridge University Press, 2007).

\bibitem{Varner}
R. L. Varner et al., ``A global nucleon-nucleus optical model potential,'' Phys. Rep. \textbf{201}, 57-119 (1991).

\end{thebibliography}

\end{document}