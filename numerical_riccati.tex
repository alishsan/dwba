\documentclass[twocolumn, prc, aps, floatfix]{revtex4-2}

\usepackage{amsmath}
\usepackage{amssymb}
\usepackage{graphicx}
\usepackage{bm}
\usepackage{hyperref}

\begin{document}

\title{Hybrid Analytic-Numerical Initialization of the Numerov Algorithm for $l>0$ Partial-Wave Scattering}

\author{Alisher Sanetullaev}
\affiliation{Department of Physics, New Uzbekistan University}

\author{Marhabo Beymamatova}
\affiliation{Department of Physics, New Uzbekistan University}

\author{Mirzabek Botirov}
\affiliation{Department of Physics, New Uzbekistan University}

\date{\today}

\begin{abstract}
The numerical solution of the radial Schr\"odinger equation for non-zero angular momentum ($l > 0$) is often hindered by the centrifugal singularity at the origin. We present a hybrid integration scheme that utilizes an analytic Riccati-Bessel initialization to satisfy initial boundary conditions for the Numerov propagator. Unlike standard power-series starts, this method incorporates the local potential depth and incident energy, preventing the excitation of spurious irregular solutions. We demonstrate that this approach preserves the discrete Wronskian to machine precision and significantly improves the convergence of phase shifts $\delta_l$ in Woods-Saxon potentials.
\end{abstract}

\maketitle

\section{Introduction}

The numerical solution of the radial Schr\"odinger equation is a cornerstone of nuclear and atomic physics. For partial waves with $l > 0$, the centrifugal barrier term, $l(l+1)/r^2$, introduces a singularity at the origin that poses challenges for standard finite-difference methods. The Numerov algorithm is widely favored for its $O(h^6)$ local truncation error and efficiency in solving second-order equations lacking first-derivative terms.

Standard implementations typically initialize the integration at $r = h$ using a simple power-series $u_l(r) \approx C r^{l+1}$. However, this approach ignores the influence of the potential $V(r)$ and energy $E$ near the origin, introducing a leading-order mismatch. This error excites the irregular solution ($r^{-l}$), leading to a drift in the Wronskian and loss of unitarity. We propose a hybrid scheme employing an analytical Riccati-Bessel start to align the numerical propagator with the physical solution from the first step.

\section{Theory and Methodology}

\subsection{The Riccati-Bessel Analytic Start}
In the region $r \to 0$, where the potential is approximately constant ($V(r) \approx -V_0$), the radial equation reduces to the spherical Bessel equation:
\begin{equation}
\left[ \frac{d^2}{dr^2} + q^2 - \frac{l(l+1)}{r^2} \right] u_l(r) = 0
\end{equation}
where $q = \sqrt{2m(E + V_0)/\hbar^2}$. The regular solution is the Riccati-Bessel function $F_l(qr)$. For $l=1$:
\begin{equation}
F_1(qr) = \frac{\sin(qr)}{qr} - \cos(qr) \approx \frac{(qr)^2}{3} - \frac{(qr)^4}{30} + \dots
\end{equation}
By using this expansion for $u(h)$, we incorporate the physical parameters $E$ and $V_0$ into the initialization.

\subsection{Discrete Wronskian Conservation}
The Numerov algorithm preserves a discrete Wronskian $W_n$:
\begin{equation}
W_n = \left(1 - \frac{h^2}{12}f_{n+1}\right) u_{n+1} v_n - \left(1 - \frac{h^2}{12}f_n\right) u_n v_{n+1}
\end{equation}
A naive start introduces a discrepancy $\Delta W = W_1 - W_0$, representing the numerical excitation of the irregular $G_l(qr)$ mode. The Bessel-start ensures $W_1 \approx W_0$, maintaining the solution on the unitary manifold.



\section{Results and Discussion}

We tested the scheme on a Woods-Saxon potential ($V_0 = 46.23$ MeV, $R = 2.0$ fm, $a_s = 0.5$ fm) for $l=1$ at $E=2$ MeV. 

\subsection{Wronskian Stability}
The power-series method exhibits an initial "shock" to the Wronskian, whereas the Bessel-start maintains $W$ to within $10^{-14}$. This stability prevents the phase-shift drift common in high-$l$ integration.

\subsection{Phase Shift Convergence}
As shown in Table \ref{tab:convergence}, the Bessel-start reduces the phase shift error by several orders of magnitude compared to the naive $r^2$ start at the same step size $h$.

\begin{table}[h]
\caption{\label{tab:convergence}Phase Shift Error Convergence $|\delta_{calc} - \delta_{exact}|$.}
\begin{ruledtabular}
\begin{tabular}{ccc}
$h$ (fm) & Naive Start Error & Bessel-Start Error \\
\colrule
0.1 & $2.4 \times 10^{-3}$ & $1.1 \times 10^{-5}$ \\
0.05 & $6.1 \times 10^{-4}$ & $6.8 \times 10^{-7}$ \\
0.01 & $2.5 \times 10^{-5}$ & $7.5 \times 10^{-10}$ \\
\end{tabular}
\end{ruledtabular}
\end{table}



\section{Conclusion}
The hybrid Bessel-Numerov scheme eliminates the initialization shock for $l > 0$ partial waves. By ensuring the initial condition is mathematically consistent with the symplectic nature of the propagator, we achieve high-precision results with larger step sizes, offering a robust foundation for modern scattering calculations.

\begin{thebibliography}{9}
\bibitem{numerov1924} B. V. Numerov, Mon. Not. R. Astron. Soc. \textbf{84}, 592 (1924).
\bibitem{abramowitz1964} M. Abramowitz and I. A. Stegun, \textit{Handbook of Mathematical Functions} (NBS, 1964).
\bibitem{thompson2009} I. J. Thompson and F. M. Nunes, \textit{Nuclear Reactions for Astrophysics} (CUP, 2009).
\bibitem{vanderraadt1983} Th. J. Van der Raadt, J. Comput. Phys. \textbf{50}, 313 (1983).
\end{thebibliography}

\end{document}
