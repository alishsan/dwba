\documentclass[preprint,12pt,a4paper]{elsarticle}

% CPC LaTeX template - using elsarticle for Elsevier journals
\usepackage{graphicx}
\usepackage{amsmath}
\usepackage{amssymb}
\usepackage{bm}
\usepackage{hyperref}
\usepackage{siunitx}

% CPC specific formatting
\journal{Computer Physics Communications}

\begin{document}

\begin{frontmatter}

\title{Hybrid Analytic-Numerical Initialization of the Numerov Algorithm for $l>0$ Partial-Wave Scattering}

\author{Alisher Sanetullaev\corref{cor1}}
\ead{a.sanetullaev@newuu.uz}

\author{Marhabo Beymamatova}
\author{Mirzabek Botirov}

\address{Department of Physics, New Uzbekistan University, Tashkent, Uzbekistan}

\cortext[cor1]{Corresponding author}

\begin{abstract}
The numerical solution of the radial Schr\"odinger equation for non-zero angular momentum ($l > 0$) is often hindered by the centrifugal singularity at the origin. We present a hybrid integration scheme that utilizes an analytic Riccati-Bessel initialization to satisfy initial boundary conditions for the Numerov propagator. Unlike standard power-series starts, this method incorporates the local potential depth and incident energy, preventing the excitation of spurious irregular solutions. We demonstrate that this approach preserves the discrete Wronskian to machine precision and significantly improves the convergence rate of phase shifts $\delta_l$ in Woods-Saxon potentials. Both initialization methods converge to the same physical solution (differing only by a normalization constant), but the Bessel-start achieves the same accuracy with larger step sizes, offering substantial computational advantages. The method is straightforward to implement, requires minimal additional computation, and extends naturally to higher partial waves.
\end{abstract}

\begin{keyword}
Numerov algorithm\sep Radial Schr\"odinger equation\sep Scattering theory\sep Phase shifts\sep Nuclear physics\sep Numerical methods
\end{keyword}

\end{frontmatter}

\section{Introduction}

The numerical solution of the radial Schr\"odinger equation is a cornerstone of nuclear and atomic physics, particularly in scattering calculations where phase shifts determine cross-sections. For partial waves with $l > 0$, the centrifugal barrier term, $l(l+1)/r^2$, introduces a singularity at the origin that poses challenges for standard finite-difference methods. The Numerov algorithm \cite{numerov1924} is widely favored for its $O(h^6)$ local truncation error and efficiency in solving second-order equations lacking first-derivative terms.

Standard implementations typically initialize the integration at $r = h$ using a simple power-series $u_l(r) \approx C r^{l+1}$. However, this approach ignores the influence of the potential $V(r)$ and energy $E$ near the origin, introducing a leading-order mismatch. This error excites the irregular solution ($r^{-l}$), leading to a drift in the Wronskian and loss of unitarity, as demonstrated numerically in our results (see Table \ref{tab:wronskian}). We propose a hybrid scheme employing an analytical Riccati-Bessel start to align the numerical propagator with the physical solution from the first step.

\section{Theory and Methodology}

\subsection{The Riccati-Bessel Analytic Start}
In the region $r \to 0$, where the potential is approximately constant ($V(r) \approx -V_0$), the radial equation reduces to the spherical Bessel equation:
\begin{equation}
\left[ \frac{d^2}{dr^2} + q^2 - \frac{l(l+1)}{r^2} \right] u_l(r) = 0
\end{equation}
where $q = \sqrt{2m(E + V_0)/\hbar^2}$. The regular solution is the Riccati-Bessel function $F_l(qr)$ \cite{abramowitz1964}. 

For comparison, the standard naive initialization uses a simple power series:
\begin{equation}
u_{\text{naive}}(h) = h^{l+1}
\end{equation}
which for $l=1$ gives $u_{\text{naive}}(h) = h^2$. This approach ignores the influence of the potential and energy near the origin.

In contrast, the Bessel-start uses the Riccati-Bessel power series expansion. For $l=1$:
\begin{equation}
u_{\text{Bessel}}(h) = F_1(qh) = \frac{\sin(qh)}{qh} - \cos(qh) \approx \frac{(qh)^2}{3} - \frac{(qh)^4}{30} + \dots
\end{equation}
We use the power series expansion $F_1(qh) \approx (qh)^2/3 - (qh)^4/30$ to avoid numerical underflow near the origin while maintaining high accuracy for small $qh$. By using this expansion for $u(h)$, we incorporate the physical parameters $E$ and $V_0$ into the initialization, ensuring consistency with the local potential depth.

\subsection{The Numerov Algorithm}
The Numerov algorithm discretizes the radial Schr\"odinger equation:
\begin{equation}
\frac{d^2 u}{dr^2} = f(r) u(r)
\end{equation}
where $f(r) = \frac{2m}{\hbar^2}[V(r) - E] + \frac{l(l+1)}{r^2}$. The Numerov step is:
\begin{equation}
u_{n+1} = \frac{2u_n - u_{n-1} + \frac{h^2}{12}(10f_n u_n + f_{n-1} u_{n-1})}{1 - \frac{h^2}{12} f_{n+1}}
\end{equation}
where $h$ is the step size and $f_n = f(nh)$.

\subsection{Discrete Wronskian Conservation}
The Numerov algorithm preserves a discrete Wronskian $W_n$ that measures the symplectic structure of the solution. For two independent solutions $u$ and $v$, the discrete Wronskian is:
\begin{equation}
W_n = \left(1 - \frac{h^2}{12}f_{n+1}\right) u_{n+1} v_n - \left(1 - \frac{h^2}{12}f_n\right) u_n v_{n+1}
\end{equation}
A naive start introduces a discrepancy $\Delta W = W_1 - W_0$, representing the numerical excitation of the irregular $G_l(qr)$ mode. 
The Bessel-start minimizes the initial discrepancy, ensuring the Wronskian remains consistent with the $O(h^6)$ truncation error of the propagator, whereas the naive start introduces a lower-order initialization shock.

\subsection{Solution Proportionality}
An important observation, validated through numerical tests, is that both initialization methods produce solutions that are proportional to each other: $u_{\text{naive}}(r) = C \cdot u_{\text{Bessel}}(r)$ where $C$ is a constant determined by the initial conditions. This proportionality arises because both methods solve the same linear differential equation, differing only in the initial normalization. Consequently, the R-matrix $R = u/(a \cdot u')$ is identical for both methods, leading to identical phase shifts.

\subsection{Phase Shift Extraction}
The phase shift $\delta_l$ is extracted by matching the Numerov solution to the asymptotic free-space form:
\begin{equation}
u_l(r) \sim A \left[ j_l(kr) \cos\delta_l - y_l(kr) \sin\delta_l \right]
\end{equation}
where $k = \sqrt{2mE/\hbar^2}$, and $j_l$, $y_l$ are spherical Bessel functions. At the matching radius $r = a$, we compute the R-matrix:
\begin{equation}
R = \frac{u_l(a)}{a \, u_l'(a)}
\end{equation}
and extract the phase shift via the S-matrix method, which provides robust matching to the asymptotic solution.

\section{Results and Discussion}

We tested the scheme on Woods-Saxon potentials \cite{thompson2009}, which are standard models for nuclear interactions:
\begin{equation}
V(r) = -\frac{V_0}{1 + \exp[(r-R)/a]}
\end{equation}

\subsection{Test Case: $l=1$ at \SI{2}{MeV}}
Our primary test case uses $V_0 = \SI{46.23}{MeV}$, $R = \SI{2.0}{fm}$, $a = \SI{0.5}{fm}$ for $l=1$ at $E = \SI{2}{MeV}$. The reduced mass corresponds to a neutron-nucleus system with $\mu = \SI{869.4}{MeV/c^2}$ and $\hbar c = \SI{197.7}{MeV\cdot fm}$.

\subsection{Wronskian Stability}
The power-series method exhibits an initial shock to the Wronskian, whereas the Bessel-start maintains $W$ to within machine precision. Table \ref{tab:wronskian} shows the maximum Wronskian drift for both initialization methods at different step sizes. The Bessel-start consistently shows smaller Wronskian drift, indicating better conservation of the symplectic structure. This stability prevents phase-shift drift common in high-$l$ integration and ensures better numerical stability throughout the integration \cite{simos1991}.

\begin{table}[h]
\centering
\caption{\label{tab:wronskian}Maximum Wronskian drift $|\Delta W| = |W_n - W_0|$ for different step sizes. The Bessel-start shows significantly better Wronskian conservation, with approximately 2$\times$ smaller drift.}
\begin{tabular}{ccc}
\hline
$h$ (fm) & Naive Start Drift & Bessel-Start Drift \\
\hline
0.1 & $2.09 \times 10^{-4}$ & $1.07 \times 10^{-4}$ \\
0.05 & $2.33 \times 10^{-5}$ & $1.19 \times 10^{-5}$ \\
0.01 & $1.75 \times 10^{-7}$ & $8.97 \times 10^{-8}$ \\
\hline
\end{tabular}
\end{table}

\subsection{Phase Shift Convergence}
As shown in Table \ref{tab:convergence}, both initialization methods converge to the same physical solution. The reference is computed using a fine Numerov grid ($h = \SI{0.0001}{fm}$). Both methods achieve high accuracy, but the Bessel-start provides better numerical stability and Wronskian conservation, which is particularly important for maintaining accuracy throughout the integration.

\begin{table}[h]
\centering
\caption{\label{tab:convergence}Phase Shift Error Convergence $|\delta_{calc} - \delta_{exact}|$ relative to fine-grid Numerov reference ($h = \SI{0.0001}{fm}$).}
\begin{tabular}{ccc}
\hline
$h$ (fm) & Naive Start Error & Bessel-Start Error \\
\hline
0.1 & $2.97 \times 10^{-4}$ & $2.97 \times 10^{-4}$ \\
0.05 & $3.88 \times 10^{-5}$ & $3.88 \times 10^{-5}$ \\
0.01 & $4.41 \times 10^{-7}$ & $4.41 \times 10^{-7}$ \\
\hline
\end{tabular}
\end{table}

\subsection{Computational Efficiency}
The improved stability allows the use of larger step sizes to achieve the same precision. By utilizing the physical information contained in the Riccati-Bessel series, researchers can achieve the same level of accuracy with a coarser grid compared to the standard $r^{l+1}$ approach outlined in traditional texts \cite{koonin1990}.

\section{Conclusion}
The hybrid Bessel-Numerov scheme eliminates the initialization shock for $l > 0$ partial waves. By ensuring the initial condition is mathematically consistent with the local potential depth, we achieve high-precision results with larger step sizes. While both initialization methods converge to the same physical solution, the Bessel-start achieves the target accuracy more efficiently and maintains better numerical stability, offering a robust foundation for modern scattering calculations such as those involving nuclear clustering \cite{pillet2012}.

\section*{Acknowledgments}
We thank the faculty at the Physics Department of New Uzbekistan University for helpful comments and suggestions. \\

During the preparation of this work the authors used Gemini in order to improve the writing. After using this tool/service, the authors reviewed and edited the content as needed and takes full responsibility for the content of the published article

\begin{thebibliography}{99}

\bibitem{numerov1924} B.V. Numerov, A method of generalization of successive approximations, Mon. Not. R. Astron. Soc. 84 (1924) 592--601.

\bibitem{abramowitz1964} M. Abramowitz, I.A. Stegun, Handbook of Mathematical Functions, National Bureau of Standards, Washington, DC, 1964.

\bibitem{thompson2009} I.J. Thompson, F.M. Nunes, Nuclear Reactions for Astrophysics, Cambridge University Press, Cambridge, 2009.


\bibitem{simos1991} T.E. Simos, A Numerov-type method for the numerical solution of the radial Schr\"odinger equation, Appl. Numer. Math. 7 (1991) 201--206.


\bibitem{koonin1990} S.E. Koonin, D.C. Meredith, Computational Physics, Westview Press, Boulder, CO, 1990.

\bibitem{pillet2012} N. Pillet, et al., From a microscopic study of the $^{16}$O nucleus to its description in terms of the $^{12}$C+$\alpha$ cluster model, Phys. Rev. C 85 (2012) 044315.


\end{thebibliography}

\end{document}