\documentclass[11pt]{letter}
\usepackage[utf8]{inputenc}
\usepackage{geometry}
\geometry{a4paper, margin=1in}
\usepackage{times}

\begin{document}

\begin{letter}{Editor-in-Chief \\
American Journal of Physics \\
American Association of Physics Teachers}

\opening{Dear Editors,}

Please accept the enclosed manuscript, \textbf{"From Transcendental Roots to Numerical Benchmarks: High-Precision Bound States of the Finite Square Well,"} for consideration in the \textit{Computational Physics} section of the \textit{American Journal of Physics}.

This paper addresses a critical pedagogical gap in the transition from undergraduate quantum mechanics to computational nuclear physics research. While the finite square well is a standard textbook problem, it is rarely treated with the numerical rigor required to serve as a benchmark for professional research tools. Students and researchers frequently utilize Distorted Wave Born Approximation (DWBA) codes with Woods-Saxon potentials, often treating these numerical solvers as "black boxes."

Our manuscript provides a pedagogical bridge between these two worlds. We present a method to validate complex Woods-Saxon solvers by utilizing the finite square well as a rigorous limit (diffuseness $a_0 \to 0$). 

The core pedagogical contribution of this work lies in the methodology. Rather than employing simple bisection or finite-difference methods, we demonstrate the implementation of a \textbf{Newton-Raphson algorithm utilizing analytical derivatives} derived from spherical Bessel function recurrence relations. This approach:
\begin{itemize}
    \item Demonstrates the superior stability and quadratic convergence of analytical over numerical derivatives.
    \item Provides students with a "Gold Standard" reference table for the $^{14}$C + n system, allowing them to quantitatively verify the integration grid and matching radii of their own codes.
    \item Illustrates the behavior of loosely bound states (such as the $1p_{1/2}$ state near threshold), where standard "shooting methods" often fail without precise derivatives.
\end{itemize}

We believe this work is ideally suited for AJP readers as it empowers educators to teach computational physics not just as code-writing, but as a discipline requiring deep analytical understanding of the underlying mathematics.

This manuscript has not been published previously and is not under consideration by any other journal. Thank you for your consideration.

\closing{Sincerely,}

\textbf{[Your Name]} \\
[Your Title] \\
[Your Institution] \\
[Your Email Address]

\end{letter}
\end{document}