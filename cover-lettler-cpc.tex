\documentclass[12pt]{letter}
\usepackage[utf8]{inputenc}
\usepackage{geometry}
\geometry{a4paper, margin=1in}

\address{Alisher Sanetullaev \\ Department of Physics \\ New Uzbekistan University \\ Tashkent, Uzbekistan \\ a.sanetullaev@newuu.uz}

\begin{document}

\begin{letter}{The Editors \\ Computer Physics Communications}

\opening{Dear Editors,}

I am pleased to submit our manuscript, titled \textbf{``Hybrid Analytic-Numerical Initialization of the Numerov Algorithm for $l>0$ Partial-Wave Scattering,''} for consideration as an original research article in \textit{Computer Physics Communications}.

The Numerov algorithm remains one of the most widely used propagators for the radial Schr\"odinger equation in nuclear, atomic, and molecular physics due to its high-order efficiency. However, for partial waves with non-zero angular momentum ($l > 0$), the standard initialization protocols often introduce numerical ``shocks'' that excite spurious irregular solutions.

In this work, we propose a hybrid integration scheme that replaces the naive power-series start with an analytic Riccati-Bessel initialization. Our primary contributions include:

\begin{enumerate}
    \item \textbf{Algorithmic Refinement:} We demonstrate how incorporating the local potential depth and incident energy into the first integration step ensures the solution remains on the unitary manifold.
    \item \textbf{Numerical Stability:} We prove that this method preserves the discrete Wronskian to machine precision ($\approx 10^{-16}$), providing a robust alternative to standard initialization for higher partial waves.
    \item \textbf{Efficiency Validation:} Using a high-precision implementation in Clojure, we show that while both methods converge to proportional wave functions, the hybrid scheme achieves target accuracies with significantly larger step sizes, offering a concrete computational advantage for large-scale scattering simulations.
\end{enumerate}

We believe this work is well-suited for the audience of \textit{Computer Physics Communications}, as it provides a straightforward yet mathematically rigorous improvement for a fundamental numerical algorithm used across multiple branches of computational physics.

This manuscript has not been published elsewhere and is not under consideration by another journal. All authors have approved the manuscript and agree with its submission.

Thank you for your time and for considering our work.

\closing{Sincerely,}

Alisher Sanetullaev

\end{letter}
\end{document}